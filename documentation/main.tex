\documentclass[author-year, prd, amsmath, amssymb, longbibliography, floatfix, reprint, superscriptaddress, a4]{revtex4-1}

\usepackage{bm}
\usepackage[toc, page]{appendix}
\usepackage[alsoload=hep]{siunitx}
\usepackage{tikzscale}
\usepackage{pgfplotstable}
\usepackage[mode=buildnew]{standalone}
\usepackage{bigints}
\usepackage{booktabs}
\usepackage{amsmath}
\usepackage{ifthen}
\usepackage{tensor}
\usepackage[hidelinks]{hyperref}
\usepackage{xfrac}
\usepackage[
	top    = 1.95cm,
	bottom = 1.9cm,
	left   = 1.5cm,
	right  = 1.5cm]
{geometry}

\definecolor{color1}{RGB}{202,0,32}
\definecolor{color2}{RGB}{244,165,130}
\definecolor{color3}{RGB}{146,197,222}
\definecolor{color4}{RGB}{5,113,176}

\hypersetup{
	colorlinks,
	linkcolor={color4},
	citecolor={color4},
	urlcolor={color4}
}

\AtBeginDocument{
	\heavyrulewidth=.08em
	\lightrulewidth=.05em
	\cmidrulewidth=.03em
	\belowrulesep=.65ex
	\belowbottomsep=0pt
	\aboverulesep=.4ex
	\abovetopsep=0pt
	\cmidrulesep=\doublerulesep
	\cmidrulekern=.5em
	\defaultaddspace=.5em
}

\pgfplotsset{compat=1.15}

\newcommand{\vect}[1]{\bm{\mathrm{#1}}}

\newcommand{\rmi}{\mathrm{i} \mkern1mu} %% roman "i"
\newcommand{\dd}{\textrm{d}}
\newcommand{\np}[1]{^{\!\!#1}}
%\newcommand{\cn}{\hspace*{-0.75pt}\mathsc{N}\hspace*{-0.75pt}}
\DeclareMathOperator{\arcsec}{arcsec}

\newcommand{\G}{\Gamma}
\newcommand{\e}{\epsilon}
\newcommand{\s}[1][]{%
	\ifthenelse{\equal{#1}{}}{S(\Theta)}{S^{#1}(\Theta)}%
}
\newcommand{\ssqrt}{\sqrt{\e (2-\e) + (1-\e)^{2}\s[2]}}
\newcommand{\eln}{\ln\!\left[\frac{2 - \e}{\e}\right]}
\newcommand{\fln}{\ln\!\left[2 - \e\right]}
\newcommand{\gln}{\ln\!\left[\e\right]}
\newcommand{\sln}{\ln\!\left[\frac{\ssqrt + (1-\e)\s}{\sqrt{\e (2-\e)}}\right]}
\newcommand{\rln}{\ln\!\left[(2-\e)-2(1-\e)\s[2]\right]}
\newcommand{\tln}{\ln\!\left[\e+2(1-\e)\s[2]\right]}
\newcommand{\lns}{\ln\!\left[\s\right]}

\newcommand*\pFqskip{8mu}
\catcode`,\active
\newcommand*\pFq{\begingroup
        \catcode`\,\active
        \def ,{\mskip\pFqskip\relax}%
        \dopFq
}
\catcode`\,12
\def\dopFq#1#2#3#4#5{%
        {}_{#1}F_{#2}\biggl(\displaystyle {\genfrac{}{}{0pt}{0}{#3}{#4}};#5\biggr)%
        \endgroup
}

\ExplSyntaxOn
\DeclareExpandableDocumentCommand{\RepQuad}{m}
{\int_compare:nT { #1 > 0 }
	{\quad \prg_replicate:nn { #1 - 1 } {\quad}}}
\ExplSyntaxOff

\newcommand{\mathsc}[1]{\mbox{\tiny{\(#1\)}}}

\renewcommand{\thetable}{\arabic{table}}

\newcommand{\dm}[1]{{\color{red}#1}}
   
\makeatletter 
	\renewcommand{\fnum@figure}{Fig.~\thefigure}
\makeatother

\makeatletter 
	\renewcommand{\fnum@table}{Table~\thetable}
\makeatother

\allowdisplaybreaks[4]

\sisetup{mode=math, range-phrase = {\text{~--~}}}

\bibliographystyle{aipauth4-1}


\begin{document}

%\preprint{AIP/123-QED}

\title[Sky tesselation]{Sky tesselation}

\author{Deyan P. Mihaylov}
\email{deyan@aei.mpg.de}

\maketitle

\section{Icosahedron}
The vertices of an icosahedron centered at the origin and with circumradius 1 are:
\begin{subequations}
    \begin{align}
    V_{1} &= (0, 0, 1) \\
    V_{2} &= (0, 0, -1) \\
    V_{3} &= \left(\frac{2}{\sqrt{5}}, 0, \frac{1}{\sqrt{5}}\right) \\
    V_{4} &= \left(-\frac{2}{\sqrt{5}}, 0, -\frac{1}{\sqrt{5}}\right) \\
    V_{5} &= \left(-\frac{5+\sqrt{5}}{10}, \sqrt{\frac{5-\sqrt{5}}{10}}, \frac{1}{\sqrt{5}}\right) \\
    V_{6} &= \left(-\frac{5+\sqrt{5}}{10}, -\sqrt{\frac{5-\sqrt{5}}{10}}, \frac{1}{\sqrt{5}}\right) \\
    V_{7} &= \left(\frac{5+\sqrt{5}}{10}, \sqrt{\frac{5-\sqrt{5}}{10}}, -\frac{1}{\sqrt{5}}\right) \\
    V_{8} &= \left(\frac{5+\sqrt{5}}{10}, -\sqrt{\frac{5-\sqrt{5}}{10}}, -\frac{1}{\sqrt{5}}\right) \\
    V_{9} &= \left(\frac{5-\sqrt{5}}{10}, \sqrt{\frac{5+\sqrt{5}}{10}}, \frac{1}{\sqrt{5}}\right) \\
    V_{10} &= \left(\frac{5-\sqrt{5}}{10}, -\sqrt{\frac{5+\sqrt{5}}{10}}, \frac{1}{\sqrt{5}}\right) \\
    V_{11} &= \left(-\frac{5-\sqrt{5}}{10}, \sqrt{\frac{5+\sqrt{5}}{10}}, -\frac{1}{\sqrt{5}}\right) \\
    V_{12} &= \left(-\frac{5-\sqrt{5}}{10}, -\sqrt{\frac{5+\sqrt{5}}{10}}, -\frac{1}{\sqrt{5}}\right)
    \end{align}
\end{subequations}

\section{Orthographic projection}

\section{Constructing finer tesselations}




\end{document}